

 \documentclass[12pt,a4paper]{report}
\usepackage{graphicx}
\usepackage{amsmath}
\begin{document}
\title{HYDROGEN ORBITALS}
\maketitle
\author{Sanmveg Saini , Durgesh Kumar}



\section{Introduction}
The Hydrogen Atom consists of a heavy ,essentially motionless proton,of charge e,together with a much lighter charged elecron 
From Coulomb's  law potential enrgy is \newline
$$V(r)=\frac{-e^2} {4\pi\epsilon r} \newline$$ 
The radial equation  is 
$$\frac{-h^2 d^2u }{2m dr^2} + \left[\frac{-e^2 }{4\pi\epsilon r} +\frac{{h^2}l(l+1)}{2m r^2}\right]u = Eu $$   \\

Our problem is to solve this equation for $u(r)$, and determine the allowed energies,E.The coulomb potential admit continuum states(E\g 0).
describing electron-proton scattering,as well as discrete bound states, representing the hydrogen atom.


\section{The Radial Wave function}

Let
 $ \kappa \equiv  \frac{-2mE}{h}   $  
for bound states E is negative and real \\
$$ \frac{d^2u}{\kappa^2dr^2} = \left[ 1- \frac{me^2}{2\pi\epsilon h^2 \kappa(\kappa r)}+ \frac{l(l+1)}{(\kappa r)^2} \right]u $$ 
This suggests that we introduce \\

$\rho \equiv \kappa r, $ and $   \rho_0 \equiv \frac{me^2}{2\pi\epsilon_0h^2\kappa}$ 
so that
$$ \frac{d^2u}{d\rho^2}=\left[1-\frac{\rho_0}{\rho} + \frac{l(l+1)}{\rho^2}\right]u$$\\
Next we examine the asymptotic form of th solutions as $ \rho \rightarrow \infty$, the constnants term in the bracket dominates,so(approx.)\\
$$ \frac{d^2u}{d\rho^2}=u$$ 
The general solution is
$$u(\rho) =Ae^{-\rho} + Be^\rho$$ 
but $e^\rho$ blows up (as $\rho\rightarrow \infty$), so $B=0$. Evidently \\
$$u(\rho)\approx Ae^{-\rho}$$\\
for large $\rho$. On the other hand as $\rho \rightarrow 0$ the centrifugal term dominates approximately then:\\
$$ \frac{d^2u}{d\rho^2}= \frac{l(l+1)}{\rho^2}u$$ 
The general solution is 
$$u(\rho)=C\rho^{l+1}+ D\rho^{-l}$$\\ 
but $\rho^{-l}$ blows up (as $\rho \rightarrow 0$), so $D=0$ . Thus \\ 
$$u(\rho)=C\rho^{l+1}$$
for small $\rho$.  
The next step is to peel off the asymptotic behaviour introducing the new function $v(\rho) $:
$$u(\rho)=\rho^{l+1}e^{-\rho}v(\rho),$$
Differentiating:
$$ \frac{du}{d\rho}=\rho^l e^{-\rho}\left[(l+1-\rho)v + \rho\frac{dv}{d\rho}\right]$$ \\ 
and 
$$\frac{d^2u}{d\rho^2}=\rho^l e^{-\rho}\left[[-2l-2+\rho+\frac{l(l+1)}{\rho}]v +2( l+1-\rho) \frac{dv}{d\rho}+\rho\frac{d^2v}{d\rho^2}\right]$$ 
In terms of $v(\rho)$, then the radial equation reads
$$\rho\frac{d^2v}{d\rho^2}+ 2(l+1-\rho)\frac{dv}{d\rho}+[\rho_0-2(l+1)]v=0$$
Finally, we assume the solution$ v(\rho) $can be expressed as a power series in $\rho$ :
$$v(\rho)=\sum_{j=0}^{\infty}c_j\rho^j$$
Differentiating term by term to detemine the coefficients: 
$$\frac{dv}{d\rho} = \sum_{j=0}^{\infty}j c_j\rho^{j-1} = \sum_{j=0}^{\infty}(j+1)c_{j+1}\rho^j$$ 
Differentiating again and adjusting the dummy index j; 
$$\frac{d^2u}{d\rho^2}=\sum_{j=0}^{\infty}j(j+1)c_{j+1}\rho^{j-1}$$
Solving;
$$\sum_{j=0}^{\infty}j(j+1)c_{j+1}\rho^j+\sum_{j=0}^{\infty}(j+1)c_{j+1}\rho^j-2\sum_{j=0}^{\infty}jc_{j}\rho^j+ [\rho_0-2(l+1)]\sum_{j=0}^{\infty}c_j\rho^j=0$$ 
Equating the coefficients of like power yields 
$$c_{j+1}=\left[\frac{2(j+l+1)-\rho_0}{(j+1)(j+2l+2)}\right]c_j$$ 
This recursion formula determines the coefficients and hence the function. \\ \\% v(\rho)% 

For large j the recursion formula can be approximated to 
$$c_{j+1} \cong \frac{2j}{j(j+1)}c_j=\frac{2}{j+1}c_j$$
Suppose for a moment that this were exact.Then
$$c_j=\frac{2^j}{j!}c_0$$ 
so 
$$v(\rho)=c_0\sum_{j=0}^{\infty}\frac{2^j}{j!}\rho^j=c_0e^{2\rho}$$
and hence 
$$u(\rho)=c_0\rho^{l+1}e^\rho$$
 which blows up at large $\rho$ because of exponential factor \\
For finding solution that we are interested in the series must terminate such that\\ 
$$c_{(j_{max} +1)}=0 $$ 
then,  $2(j_max +l+1)-\rho_0=0$ 
defining
$$n\equiv j_max +l+1$$ 
n = principal quantum number\\
$\rho_0=2n$\\ \\
so the allowed energies 
$$E=\frac{-h^2\kappa^2}{2m}=\frac{me^4}{8\pi^4\epsilon_0^4 h^2 \rho_0^4}$$
this is the bohr formula.\\ \\
Above equations give $$\kappa=\left[\frac{me^2}{4\pi\epsilon_0 h^2}\right]\frac{1}{an}=\frac{1}{an}$$\\
where $$\text{Bohor's Radius = }a\equiv\frac{4\pi\epsilon_0 h^2}{me^2}=0.529 \times 10^{-10}$$
It follows that 
$\rho =\frac{r}{an}$
and the spatial wave function for hydrogen are labelled by three quantum numbers (n,l,m)
$$\psi_{nlm}(r,\theta,\phi)=R_{nl}(r)Y^m_l(\theta,\phi)$$  \\
where 
$$R_{nl}(r)=\frac{\rho^{l+1}e^{-\rho}v(\rho)}{r}$$
and $v(\rho)$  is a polynomial of degree $ j_{max}=n-l-1 $ whose coefficeints are determined by the recursion formula.
$$c_{j+1}=\left[\frac{2(j+l+1-n)}{(j+1)(j+2l+2)}\right]c_j$$
The ground state (n=1) is 
$E_1=-13.6eV$.
Evidently the binding energy is 13.6 eV.\\\\
For hydrogen,  after limitaions l=0,m=0 ,so,
$\psi_{100}(r,\theta,\phi)=R_{10}(r)Y^0_0(\theta,\phi)$  \\
The recursion formula truncates after the first term so $v(\rho) $is a constant($c_0$), and \\
$$R_{10}(r)=\frac{c_0}{a}e^{-r/a}$$ 
After normalizing it we get $c_0 =2\times \sqrt{a}$ and  $Y^{0}_0 =1/{\sqrt{4\pi}}$\\
Therefore, the ground state of hydrogen is 

$$\psi_{100}(r,\theta,\phi)=\frac{1}{\sqrt{\pi}}e^{-r/a}$$
For arbitray n, the possible values of $l$ are
 $l=0,1,2,\ldots,n-1$
and for each $l$ there are $(2l+1)$ possible values of $m$,so the total degeneracy of the energy level $E_n$ is
$d(n)=\sum_{l=0}{n-1}=n^2$.
The formula for $v(\rho)$ is a function well known to applied mathematician; apart from normalization ,it can be written as
$$v(\rho)=L^{2l+1}_{n-l-1}(2\rho)$$
where 
$$L^{p}_{q-p}(x) \equiv (-1)^p\frac{d}{dx}^P L_q(x)$$
is an associated Laguerre polynomial, and
$$ L_q(x)=e^x{\frac{d}{dx}}^q (e^{-x}x^q)$$
is the $q^{th}$ Laguerre Polynomial.The normalized hydrogen wavefunctions are 
$$\psi=\sqrt{\left(\frac{2}{na}\right)^3 {\frac{(n-l-1)!}{2n((n+1)!)^3}}}
 \times e^{-r/na}\times{\frac{2r}{na}}^l\times  \left(L^{2l+1}_{n-l-1}\times\frac{2r}{na}\right)\times Y^{m}_l(\theta,\phi)$$ 


This is one of the very few realistic systems that can be solved at all in exact closed form.The wave function de[end on all three
quantum numbers whereas the energies are determined by n alone.\\
\end{document}
